\documentclass[11pt]{article}


\usepackage{res/config/cpi}

\geometry{letterpaper}

\newcommand*\Title{Proposta di riprogettazione della\\Dashboard "Situazione Italia"\\del Dipartimento della Protezione Civile}
\newcommand*\cpiType{CPI Template}
\newcommand*\Date{June 2011}
\newcommand*\Author{Michele Perlino}
\title{Proposta di riprogettazione della\\Dashboard "Situazione Italia"\\del Dipartimento della Protezione Civile}
\author{Andrew Hobbs}
\date{\today}
%-----------------------------------------------------------

\usepackage{res/config/cpi} % This is what makes your document look like a cpi document.


\begin{document}

\begin{titlepage}
\maketitle
\end{titlepage}

\linespread{1.15} %Set standard document linespacing

% Abstract
\subfile{res/section/0-abstract}

{
    \hypersetup{linkcolor=black}
    \rmfamily{\tableofcontents}
}
\clearpage


% Sezione 1: Contesto di riferimento
\subfile{res/section/1-contesto-riferimento.tex}


% Sezione 2: Considerazioni sulla dashboard del DPC
\subfile{res/section/2-considerazioni-dashboard-DPC.tex}


% Sezione 3: Riprogettazione proposta
\subfile{res/section/3-riprogettazione-proposta.tex}


% Sezione 4: Conclusioni
\subfile{res/section/4-conclusioni.tex}



% \bibliography{codes_impact}
% \bibliographystyle{plainnat}

\end{document}