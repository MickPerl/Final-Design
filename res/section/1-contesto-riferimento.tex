\documentclass[../../main.tex]{subfiles}

\begin{document}
\section{Contesto di riferimento}
Il 2020 passerà alla storia come uno dei periodi più complicati su numerosi fronti, da quello sanitario a quello sociale, da quello politico a quello economico: l'11 marzo 2020, l'Organizzazione Mondiale della Sanità (OMS) ha dichiarato che il focolaio internazionale di infezione da nuovo coronavirus SARS-CoV-2 potesse essere considerato una pandemia.

Relativamente alle misure di contrasto, sicuramente la medicina e la ricerca hanno e stanno svolgendo un ruolo primario, tuttavia molto importanti sono anche i media e i vari canali informativi, in quanto responsabili della comprensione e consapevolizzazione dei cittadini a tal proposito.\\
In questo ambito, uno degli strumenti tecnologici di maggior autorevolezza e adozione è la dashboard ufficiale del Dipartimento della Protezione Civile (\href{https://opendatadpc.maps.arcgis.com/apps/opsdashboard/index.html#/b0c68bce2cce478eaac82fe38d4138b1}{COVID-19 Situazione Italia}). Essa sta svolgendo un ruolo di informazione pubblica di assoluta rilevanza, tuttavia abbiamo individuato alcune criticità che, qualora risolte, concorrerebbero a migliorarne l'esperienza utente e l'usabilità, nonché la capacità informativa.\\
Al fine di dar sostanza a queste nostre osservazioni, ci siamo permessi di riprogettarne l'interfaccia e, nel report corrente, descriviamo le riflessioni e gli studi condotti in ogni fase.\\
Sottolineiamo che abbiamo focalizzato la riprogettazione su un segmento di utenza specifico, che è un anello essenziale nella valorizzazione del diritto all'informazione dei cittadini: trattasi dei giornalisti impegnati nella redazione di articoli volti a comunicare i dati epidemiologiche.
Crediamo che una dashboard orientata a loro debba presentare visualizzazioni dall'alto impatto informativo, nei termini di una contestualizzazione puntuale e completa delle variabili coinvolte entro il periodo temporale e la circostanza spaziale in cui si manifestano: reputiamo essere questa la direzione privilegiata e più spedita verso una comprensione profonda del quadro epidemiologico, cui possa aver seguito una comunicazione ai cittadini tanto completa quanto fruibile.
\end{document}